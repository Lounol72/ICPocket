\documentclass[12pt,a4paper, twoside]{article}
\usepackage[T1]{fontenc}
\usepackage{fancyhdr}
\usepackage[utf8]{inputenc}
\usepackage[french]{babel}
\usepackage{color}
\usepackage{graphicx}
\usepackage{hyperref}
\usepackage[left=3cm,right=3cm,top=2cm,bottom=2cm]{geometry}
\usepackage{array}
\pagestyle{fancy}
\setlength{\headheight}{12pt}
\usepackage[acronym]{glossaries}
\makeglossaries
\begin{document}

\begin{titlepage}
    \begin{minipage}[t]{0.48\textwidth}
        \includegraphics[height=1.01cm]{logolemansU.png}
    \end{minipage}
    \hfill
    \begin{minipage}[t]{0.25\textwidth}
        \includegraphics[height=1.6cm]{logo_IC2.png}
    \end{minipage}
    
    \vspace{2cm}
    \begin{center}
        \Large\textbf{Le Mans Université}\\
        \vspace{0.5cm}
        Licence Informatique 2ème année\\
        Module 17UF02 Rapport de Projet\\
        \vspace{0.5cm}
        \Large\textbf{Titre du projet}\\
        \vspace{1cm}
        {\large Noms des auteurs}\\
        \vspace{0.5cm}
        {\normalsize \today} 
    \end{center}
\end{titlepage}

\newpage
\tableofcontents
\newpage
\abstract{
    Ceci est le texte de mon résumé...
}
\section{Introduction}

\emph{Cette introduction présentera le sujet qui sera traité et le travail avec une présentation du plan adopté}\\
Dans le cadre d'un projet de notre fin d'année de L2 informatique, le type de jeu vidéo créer est un jeu de rôle qui se joue en tour par tour en faisant affronter des créatures fantastiques a l’instar du jeu phare de Nintendo nommé « Pokemon ».\\
Ce projet a été réaliser en langage C avec la librairie SDL et contrairement a son inspiration , le jeu ne se passe pas dans un pays tous droit sorti de l’imaginaire d’une personne mais bel et bien au sein du bâtiment de l’Institut Claude Chappe, Vous incarnerez un nouvel étudiant rentrant en première année et affronterez d'autres étudiants et enseignants chercheurs déjà installés dans le bâtiment a travers des duels de ICmon.\\ 

En première partie, nous allons expliquer comment nous avons conçu le jeu, une présentation général et détaillé de l’univers dans lequel il se situe, son histoire ainsi que celui du joueur communément appeler dans le jargon son « lore » .
Ensuite sa direction artistique c’est a dire tous l’aspect graphique seront expliciter , avec quels logiciels les sprites des Icmons, du joueur et des niveaux ont été esquisser et enfin toutes les possibilités offertes au joueur pendant son aventure.\\

Puis nous parlerons de l’organisation, quelles missions a été confié pour chaque personne du groupe et sous quels deadline devaient-elles être rendu ,tous ceci a été planifié grâce a un diagramme de Gantt que nous avons réalisé.\\

En troisième partie nous expliquerons la gestion des Icmons , comment ils ont été générés et comment la gestion des duels a été réalisé, puis nous parlerons de la gestion des bases de données, comment les données des Icmons et le jeu ont été sauvegardé puis chargé et ensuite réutiliser de nouveau.\\

Puis nous aborderons la gestion de la carte, comment le joueur se déplace dans l'espace et comment la caméra le suit. et lorsqu'il rencontre un obstacle, une collision doit se produire.\\

Et enfin nous mettre en évidence la rendu du jeu , les sprites que nous avons créer doivent être implémenté directement dans le code, le système de combat c’est a dire les points de vie décrémenter lorsqu’une attaque est subie ,devrait être cohérent avec ce que le joueur voit sur l’écran et pour finir l’interface par quel moyen l’interface du jeu a été conçu, de quelle façon le joueur change de fenêtre lorsque tel bouton est appuyé\\\\
\lfoot\fancyfoot{\footnotesize *Pokemon:le jeu sera présenter dans la suite du rapport\\*Sprite: image 2D utilisé dans les jeux vidéos\\*ICmon: contraction de l'expression "Institut Claude chappe " et "Monstre"\\*Lore: histoire d'un jeu vidéo\\*deadline: date limite\\}
\newpage
\section{Conception}
    Dans cette première partie nous allons présenter .... . La figure 1 illustre .....

    L’instruction \/begin figure [!h] force l’image à se positionner comme on le désire. Le positionnement de l’image dépend également de ses dimensions, si elle est située en bas de page, et qu’elle est trop grande, elle va se positionner sur la page suivante et le texte qui suit se retrouvera donc avant l’image. Il faudra réduire la taille. Ceci peut se faire avec l’option scale de includegraphics.


\begin{figure}[h]
    \centering
    \includegraphics[width=1\textwidth]{image.png}
    \label{fig:logo}
\end{figure}
\subsection{Présentation du jeu}
\subsection{Direction artistique}
\subsection{Fonctionnalités}
\section{Organisation}
\subsection{}
\subsection{Sous partie 2}
\section{Partie 2-Les tableaux en Latex}

    Les algorithmes de \LaTeX pour mettre en forme les tableaux ont quelques imperfections. L’une d’entre elles est qu’il n’affiche pas automatiquement le texte sur plusieurs lignes dans les cellules, même si celui-ci déborde de la largeur de la page. Pour les colonnes qui contiendront une certaine quantité de texte, il est recommandé d’employer l’attribut p et d’indiquer la largeur désirée de la colonne (bien que cela puisse obliger à effectuer quelques ajustements avant d’obtenir le résultat souhaité) (figure2etfigure3oufigure 4)[3] [3]. ...

    Afin de mener à bien ce projet, nous avons suivi un planning prévisionnel qui nous a permis de nous organiser et de nous assurer que nous respections les dates limites. Nous avons donc dû faire un diagramme de Gantt pour nous aider à planifier nos tâches.
    \begin{figure}[h]
        \centering
        \includegraphics[width=0.8\textwidth]{../../assets/Title Screen/BG.jpg}
        \label{fig:GANTT}
        \caption{Diagramme de Gantt}
    \end{figure}
\subsection{Planning Prévisionnel}
\begin{table}[h]
    \centering
    \begin{tabular}{|c|c|c|}
        \hline
        Un titre de colonne & Un autre titre & Encore un \\
        \hline
        à gauche & au centre & à droite \\
        \hline
        d & au centre & à droite \\
        \hline
        Si j’écris plus de texte je dépasse la taille de la page & texte & texte \\
        \hline
    \end{tabular}
    \vspace{1cm}
    \caption{\emph{Un tableau simple}}
    \label{tab:test}
\end{table}

\subsection{Répartition des tâches}
\subsection{Outils}
\newpage
\begin{center}
\begin{tabular}{|c|p{8cm}|}
\hline
2 - 1 & On met un grand texte qui sera sur plusieurs lignes \\
\hline
\end{tabular}
\end{center}

\section{Developpement}
\subsection{Gestions des ICMons}
    \subsubsection{structures de données}
    \subsubsection{systèmes de combats}

\subsection{Gestion des bases de données}
    \subsubsection{sauvegarde}
    \subsubsection{base de données des icmons}

\subsection{Gestion de la map}
    \subsubsection{chargement de la map}
    \subsubsection{caméra}
    \subsubsection{deplacement}

\subsection{rendu du jeu}
    \subsubsection{gestion de l'interface}
    \subsubsection{gestion des combats avec SDL}


\section{Bilan et résultats}
\section{Références}
\section{Annexes}
\end{document}